%%%%%%%%%%%%%%%%%%%%%%%%%%%%%%%%%%%%%%%%%%%%%%%%%%%%%%%%%%%%%%%%%%
%%%%%%%%%%%%%%%%%%%%%%%%%%%%%%%%%%%%%%%%%%%%%%%%%%%%%%%%%%%%%%%%%%
%%
%%		CUMC 2021 
%%		Beamer Workshop
%%		August 7th, 2021
%%		Emerson Doyle, Huron University College
%%		Contact the author at:  edoyle9@uwo.ca
%%
%%%%%%%%%%%%%%%%%%%%%%%%%%%%%%%%%%%%%%%%%%%%%%%%%%%%%%%%%%%%%%%%%%
%%%%%%%%%%%%%%%%%%%%%%%%%%%%%%%%%%%%%%%%%%%%%%%%%%%%%%%%%%%%%%%%%%
%%
%%		Presentation Template---distributed at workshop
%%		Please feel free to utilize, modify, and distribute this template as you like
%%		
%%
%%		NOTE: to compile this file immediately, you'll have to scroll down and comment out any \includegraphics commands
%%		Alternatively you can add six jpeg files to the folder where you put this file, named image1.jpeg, image2.jpeg, etc.
%%
%%%%%%%%%%%%%%%%%%%%%%%%%%%%%%%%%%%%%%%%%%%%%%%%%%%%%%%%%%%%%%%%%%
%%%%%%%%%%%%%%%%%%%%%%%%%%%%%%%%%%%%%%%%%%%%%%%%%%%%%%%%%%%%%%%%%%

%%		We need to start by telling TeX that we're making a presentation. 

\documentclass{beamer}	

%%		Note the Beamer class should automagically load a bunch of packages and style files, including amsmath, amsfonts, amssymb, amstext, 
%%		graphicx, geometry, xcolor, hyperref, and url. If you need something extra like fitch or tikz, invoke them here.	
				

%% 		Next we have a bunch of commands that format the presentation to a visual style I personally prefer.

\usetheme{Darmstadt}						%my favourite theme---list of other default options included with Beamer class at 
										%https://www.overleaf.com/learn/latex/Beamer#Reference_guide
										%if you comment out \usetheme{} altogether, Beamer defaults to a very plain look
											
%\usecolortheme{}							%this can be used to change the default colours of a particular theme
										%again, see https://www.overleaf.com/learn/latex/Beamer#Reference_guide for details
										%I like the default colours of Darmstadt, so I just leave this out
											
\setbeamertemplate{navigation symbols}{}			%I don't find the navigation symbols to be useful. comment this out to include the symbols on all slides

%\setbeamertemplate{footline}[frame number]		%uncomment this to include an indication of the Current/Total Slide # at the bottom right of each slide
										%I prefer to just use a theme (like Darmstadt) that builds this into the look somehow
											
%%		Beamer defaults to what I think are odd choices for fonts, so the next few commands clean that up a little
										
\usepackage{lmodern}						%makes Beamer use a modified Computer Modern font, which ``fixes'' certain odd character choices
\usefonttheme{professionalfonts}				%makes Beamer display mathematics in a more typical LaTeX way (serif, italic style)
\usefonttheme{structurebold}					%makes Beamer display titles and other ``structural'' text in a bolded typeface

\renewcommand{\qedsymbol}{$\blacksquare$}		%Beamer defaults to an open square to indicate the end of a proof, which is just... just wrong.


%%		That's it for the preamble! Now the info to create the Title Page for your talk. This won't actually print anything until you create a frame and use the 
%%		\titlepage command, but it puts the information on-deck in the right way. You can comment out any of these fields if you don't want them.


\title{presentation title}						%this can take a parameter for a short title which appears in various places depending
										%on the theme chosen:	\title[short title]{presentation title}
\subtitle{presentation subtitle}
\institute{your institutional affiliation}
\author{your name}
\date{the date}								%write in a date, or if compiled on the day-of just use:	\date{\today}

%%		For the \institute argument, I like to include a bit more information, including my email address. \texttt makes the email address appear
%%		in a typewriter font to distinguish it well.

%\institute{Your Institution\\Your Department\\ \href{mailto:youremail@domain.ca}{\texttt youremail@domain.ca}}


%%		Finally ready to actually create some content! As usual for a LaTeX document, we need to tell TeX to begin/end the document.
%%		After that, Beamer presentations are divided into Frames. A frame is kind of like one full slide of content (one ``page''), although we can
%%		use overlays to ``build-up'' content in a frame piece by piece when displayed as a pdf. 

%%		The first Frame should almost always be your title page. 

\begin{document}


\begin{frame}[plain]							%the [plain] parameter tells Beamer to not display static structural elements from the current theme
										%being used, which would just clutter the slide. You can delete it if you'd like
	\titlepage
	
\end{frame}


%%		Some people like to include a Table of Contents next as a way to walk the audience through the structure of their talk. This necessitates having 
%%		good Section/Sub-Section structure, which makes for an organized talk!

\begin{frame}[plain]

	\tableofcontents							%the [pausesections] argument here builds each section out one at a time on the slide if you want to
										%emphasize each section in your summary a little more
	
\end{frame}



%%		Below are a few basic content Frames that give you an idea of how to organize your slides and overall presentation using Beamer. 
%%		Obviously everything will be dependent upon your particular talk and style preferences. Mix-and-match as necessary!
%%		You can consult the massive Beamer Users Guide at http://tug.ctan.org/macros/latex/contrib/beamer/doc/beameruserguide.pdf for details.


\section{Frame Basics}						%this will appear in the Table of Contents
\subsection{Simple Frames \& Formatting}			%so will this

\begin{frame}{Frame Title: Our First Frame}		%the Frame Title argument is optional
	
	Beamer will lay out your text in a generally pleasant way if you just type some basic stuff on a slide.
	
	\bigskip								%alternatively use \smallskip or \medskip
	
	Leaving a space between text elements will not generate space on the slide (as is typical of \LaTeX). So using the \alert{skip} commands or \alert{vfill} generates pleasant spacing as necessary.
	
	%\bigskip
	\vfill									%in Beamer this doesn't put what's below right at the bottom, rather it pads in a more modest way
										%I think Beamer cuts the page up into thirds, but I'm not entirely confident about that 
	
	As you can see the \alert{alert} command is unique to Beamer, allowing you to highlight a text element in a more substantial way than an \emph{emphasize} or \textbf{bold} would. Perhaps better for presentations.
	
\end{frame}



\begin{frame}{Using Columns for Text}

	You can also create \alert{columns} of various sizes to help organize your text, as demonstrated below:
	
	\vfill
	
	\begin{columns}
		\column{0.5\textwidth}				%you can give absolute lengths in cm or inches, columns can also be different widths
		This is the first column, with a width specified as half of the overall text-width. 
		
		\medskip
		
		It appears on the left side of the page.
		
		\column{0.5\textwidth}
		This is the second column, with the same width.
		
		\medskip
		
		It appears on the right side of the page.
	\end{columns}

	\vfill

	You can see that there's no need to partition the entire Frame into columns if you don't want to.

\end{frame}



\begin{frame}{Better Columns \& Centering}

	That last slide was super-ugly however. We can make it look nicer by shrinking the column size a bit (so text boundaries are inside the boundaries of normal text) and aligning the columns at the \alert{top}:
	
	\vfill
	
	\begin{columns}[t]						%the [t] parameter on each column is doing the work here, we also have [b], [c]
		\column{0.4\textwidth}							
		Now the text of each column will start in the same place. 
		
		\medskip
		
		No matter whether the columns are the same length.
		
		\column{0.4\textwidth}
		This is the second column, with the same width.
		
		\medskip
		
		It appears on the right side of the page.
	\end{columns}

	\vfill
	
	\begin{center}
		There! Much better.\\ Note the \alert{center} environment also works in Beamer.
	\end{center}

\end{frame}


\subsection{List Environments}

\begin{frame}{Lists}

	All the usual \LaTeX\ environments for lists work:		%the \ after the command here creates a single space
	
	\vfill
	
	\begin{columns}[c]							%here centering the columns vertically with [c] looks nice
		\column{0.3\textwidth}
		\textbf{Itemize:}
		\begin{itemize}
			\item First item
			\item Second item
			\item Third item
		\end{itemize}
		
		\column{0.3\textwidth}
		\textbf{Enumerate:}
		\begin{enumerate}
			\item First enum
			\item Second enum
			\item Third enum
		\end{enumerate}
		
		\column{0.3\textwidth}
		\textbf{With custom parameters:}
		\begin{enumerate}						%we can create custom labels in \itemize too
			\item[(i)] First item
			\item[(ii)] Second item
			\item[(iii)] Third item
		\end{enumerate}
	\end{columns}
	
	\vfill
	
	Different themes will create different styles of bullets.
	
\end{frame}


\begin{frame}{The Description Environment}
	
	A really handy environment when you want to list a bunch of definitions or terms is \alert{Description}:
	
	\begin{description}
		\item[Batman:] a superhero
		\item[Hela:] a supervillain
		\item[Aquaman:] a joke
	\end{description}
	
\end{frame}


\subsection{Font Sizes \& Content Quantity}

\begin{frame}{Font Sizing}

	We can use relative sizes for fonts just like in regular \LaTeX:
	
	\begin{columns}[c]
	\column{0.25\textwidth}
	\begin{itemize}
		\item {\tiny tiny}							
		\item {\scriptsize scriptsize}
		\item {\footnotesize footnotesize}
		\item {\small small}
		\item {\normalsize normalsize}
		\item {\large large}
		\item {\Large Large}
		\item {\LARGE LARGE}
		\item {\huge huge}
		\item {\Huge Huge}
	\end{itemize}
	
	\column{0.6\textwidth}
	I'd shy away from exploiting font sizes to try to fit more content in one Frame, as that's bad presentation design. 
	
	\medskip
	
	Rather, try to think ahead and include only the most relevant points on the slide as \alert{prompts} for you to elaborate upon during the presentation.
	
	\medskip
	
	Even with proofs---leave out rote calculations, and try to break the argument into conceptual steps that each get their own slide(s).
	
	{\tiny (This is clearly too much text for one slide!)}
	\end{columns}

\end{frame}


\section{Blocks \& Mathematics}
\subsection{Working with Blocks}


\begin{frame}[plain]							%if you want to remind your audience of where they are in the presentation use a TOC frame again
	\tableofcontents[current]					%the [current] parameter greys out the sections before and after the current place in the presentation
\end{frame}


\begin{frame}{Using the Block Environment}
	
	\begin{block}{A Basic Block}
		This is a basic block. Its appearance will depend heavily upon the theme and colortheme you chose.
	\end{block}
	
	\begin{block}{}							%here we've left the title argument empty
		This is a block without a title.
	\end{block}
	
	\vfill
	
	\begin{columns}[c]
	\column{0.4\textwidth}
	You can use \alert{blocks} in columns. 
	
	\medskip
	
	And you can include lists inside of blocks.
	
	\medskip
	
	Or pretty much anything else you want too.
	
	\column{0.4\textwidth}
	\begin{block}{Block in Column}
		\begin{itemize}
			\item This
			\item is a
			\item list
		\end{itemize}
	\end{block}
	\end{columns}
	
\end{frame}


\begin{frame}{Other Types of Blocks}

	\begin{alertblock}{An Alert Block}				%again, title is optional---you can leave it empty
		Again, the precise colour and style of this is determined by the theme and colortheme you've chosen.	
	\end{alertblock}
	
	\begin{definition}
		A definition block doesn't take a title argument. I think it will iterate TeX's internal definition numbering for the file as well.
	\end{definition}
	
	\begin{example}
		An example of an example block. Again, no title argument.
	\end{example}
	
\end{frame}


\subsection{Mathematical Environments}


\begin{frame}{Mathematical Blocks}
	
	\begin{theorem}[Theorem Name]
		You still need to enclose mathematical symbols in `\$' as with normal \LaTeX. Like this: $a^2 + b^2 = c^2$
	\end{theorem}
	
	\begin{corollary}[Corollary Name]
		But notice that Beamer puts everything inside these boxes in italics. \alert{usefonttheme\{professionalfonts\}} from the preamble makes everything inside math-mode look like normal \LaTeX.
	\end{corollary}
	
	\begin{proof}
		The proof block includes a tombstone at the end. Recall we modified it to a black square in the preamble, because it's much more definitive and true-looking than a simple open rectangle.
	\end{proof}
	
\end{frame}


\begin{frame}{An Example!}

	\begin{theorem}
		There is no largest prime number.
	\end{theorem}
	
	\begin{proof}
		\begin{enumerate}
			\item Suppose $p$ were the largest prime number.
			\item Let $q$ be the product of the first $p$ numbers.
			\item Then $q + 1$ is not divisible by any of them.
			\item But $q + 1$ is greater than $1$, thus divisible by some prime
					number not in the first $p$ numbers.\qedhere	
												%we use \qedhere as otherwise the proof block puts it on its own line, wasting space
		\end{enumerate}
	\end{proof}
	
	The proof used \textit{reductio ad absurdum}.		%This example is adapted from section 3 of the Beamer User Guide
											%http://tug.ctan.org/macros/latex/contrib/beamer/doc/beameruserguide.pdf
			
\end{frame}


\begin{frame}{Typical Mathematical Environments}
	
	Instead of presenting all of this using the special Beamer blocks, you can just use regular mathematical environments from \LaTeX.
	
	\begin{equation}						%{equation*} suppresses showing the equation number on the right
		E = mc^2
	\end{equation}
	
	\[ We're in math-mode now!\]
	
	\bigskip
	
	Single dollar signs around some text allow for in-line math-mode as normal: $\pi = 3$.
	
\end{frame}


\begin{frame}{Some Rough Algebra, Aligned!}

	\begin{theorem}
		For all natural numbers $n\geq5$, $2^n>n^2$
	\end{theorem}
	
	\vfill
	
	\footnotesize{
	{\bf Basis Clause:} When $n=5$, we have that $2^5 = 32 > 25 = 5^2$. 
	
	\medskip
		
	{\bf Inductive Step:} For each $n\geq5$ up to some arbitrary $n$, if $2^n > n^2$, then $2^{n+1} > (n+1)^2$. So we have\dots
		\begin{align*}						%again, the * here suppresses the enumeration of the lines on the right
			2^{n+1} &= 2(2^n)\\
			&> 2n^2\qquad\qquad\qquad (\textrm{by inductive hypothesis})\\
			&= n^2 + n^2\\
			&\geq n^2 + 5n\qquad\qquad (\textrm{Since } n\geq5)\\
			&= n^2 + 2n + 3n\\
			&> n^2 + 2n + 1 = (n + 1)^2\qed
		\end{align*}}						
										%this is too much rote algebra for one slide, as is obvious by me needing to shrink the font size

\end{frame}

\subsection{Including Code}

\begin{frame}[fragile]{The Semiverbatim Environment}

	If you need to include code in your presentation, you can use the \alert{semiverbatim} environment:
	
	\begin{semiverbatim}
		\\begin\{frame\}[fragile]\{The Semiverbatim Environment\}			
		If you need to include code in your presentation, you can 	
		use the \\alert\{semiverbatim\} environment:
		\\begin\{verbatim\}
		\\end\{verbatim\}
			This is rather a self-referential frame.
		\\end\{frame\}
	\end{semiverbatim}
	
	This is rather a self-referential frame.
	
	\medskip
	
	Remember to add a backlash in front of all $\backslash$, \{, \} characters while in the \alert{semiverbatim} environment, so \TeX\ knows to print them rather than interpret them.

\end{frame}


\section{Working with Objects}
\subsection{Graphics \& Figures}


\begin{frame}								%no title on this frame---that's okay!

	Images can be inserted as normal, using \alert{includegraphics} with all the usual parameters:
	
	{\centering							%Beamer handles image centering a little funny. the Center environment adds whitespace both before
	\includegraphics[scale=0.5]{image1}\par		%and after the image, which is annoying. the \centering command here does not, but be careful to
	}									%enclose the content to be centered in braces and force a paragraph break at the end with \par
	
	It's easiest to just make sure that the image file is in the same folder as the presentation file.
	

\end{frame}



\begin{frame}

	\begin{columns}
	\column{0.5\textwidth}
		We can also include images in a columns environment to create more typical slide layouts.
		
	\column{0.5\textwidth}
	
	\centering										%I didn't include the braces here because nothing comes after the image, so I don't need to
	\includegraphics[width=\columnwidth]{image2}			%``end'' the centering for the rest of the frame
	\end{columns}
	
\end{frame}



\begin{frame}[plain]									%adding the [plain] parameter just gives us a full-page image if your image is big enough!
												%otherwise just scale it appropriately
	\centering
	\includegraphics[height=\textheight]{image3}
	
\end{frame}



\begin{frame}{Figures are Fun!}

	\begin{columns}
	\column{0.5\textwidth}
		The \alert{Figure} environment works just the same as well.
		
	\column{0.5\textwidth}
	
	\centering
	\begin{figure}
		\includegraphics[width=\columnwidth]{image4}
		\caption{Ironic that \alert{\textit{Bat}}man can't fly.}
	\end{figure}
	
	\end{columns}
	
\end{frame}


\subsection{The Tabular Environment}


\begin{frame}{Inserting Tables}
	
	Again, the \alert{Tabular} environment works just like it does in \LaTeX:
	
	\begin{center}
	\begin{tabular}{r | c | c | c |}
		& Powerpoint & Keynote & Beamer \\
		\hline\hline
		Snazzy Looking & \textcolor{red}{$\times$}  & \textcolor{teal}{\checkmark} & \textcolor{teal}{\checkmark} \\
		\hline
		Handles Math & \textcolor{teal}{\checkmark} & \textcolor{red}{$\times$} & \textcolor{teal}{\checkmark} \\
		\hline
		Does What You Expect & \textcolor{red}{$\times$} & \textcolor{teal}{\checkmark} & \textcolor{teal}{\checkmark} \\
		\hline
		Easy to Use & \textcolor{red}{$\times$} & \textcolor{teal}{\checkmark} & \textcolor{teal}{\checkmark} \\
		\hline
		Makes Work More True & \textcolor{red}{$\times$} & \textcolor{red}{$\times$} & \textcolor{teal}{\checkmark}\\
		\hline
	\end{tabular}
	\end{center}
	
	So does \textcolor{magenta}{textcolor}, obviously!
	
\end{frame}


\section{Overlays \& Transitions}
\subsection{Stepping Through Your Content}


\begin{frame}{The Pause Command}

	For pretty much any reveal\dots\ \pause you can use the \alert{pause} command! \pause
	
	\vfill
	
	It also works within environments, like \alert{Itemize:} \pause
	
	\begin{itemize}
		\item First item \pause
		\item Second item \pause
		\item Third item
	\end{itemize}
	
	\vfill
	
	Beamer takes a single frame and break it up into multiple ``slides'' when it outputs to pdf. So you're just scrolling through the pdf deck as normal, but the items will ``build'' on the frame.
	
\end{frame}



\begin{frame}{The Pause Command}

	It works just as well for images: \pause
	
	\begin{center}
	\includegraphics[scale=0.25]{image5}
	\end{center}	\pause
	
	The ``bubbles'' up top in a theme like Darmstadt won't advance until we get to the next Frame.

\end{frame}


\subsection{More Advanced Methods}


\begin{frame}{Other Ways to Accomplish the Same Thing}

	If we have a list, we can include the parameter $[<+->]$ when we begin the listing environment: 

	\begin{itemize}[<+->]
		\item First item
		\item Second item
		\item Third item
	\end{itemize}
	
	\vfill
	
	{\scriptsize (take a look at the code to see how it works!)}

\end{frame}


\begin{frame}{Other Ways to Accomplish the Same Thing II}

	We can also get more fine-grained with these parameters on individual items:

	\begin{itemize}
		\item<2,4> First item (shows only on second and fourth steps)
		\item<1-> Second item (shows throughout frame)
		\item<3-4> Third item (shows on third and fourth step)
		\item<4-> Fourth item (appears on fourth step, remains)
		\item<5-> Fifth item (shows on last step)
	\end{itemize}
	
	\vfill
	
	{\scriptsize (take a look at the code to see how it works!)}

\end{frame}


\begin{frame}{Other Ways to Accomplish the Same Thing III}

	Using the \alert{setbeamercovered} command with the argument \alert{transparent} lets us see the other items greyed-out:
	
	\setbeamercovered{transparent}

	\begin{itemize}
		\item<2,4> First item (shows only on second and fourth steps)
		\item<1-> Second item (shows throughout frame)
		\item<3-4> Third item (shows on third and fourth step)
		\item<4-> Fourth item (appears on fourth step, remains)
		\item<5-> Fifth item (shows on last step)
	\end{itemize} 
	
	\setbeamercovered{invisible}
	
	\uncover<6>{Make sure to set \alert{setbeamercovered} back to \alert{invisible} unless you want all overlays in the whole presentation to work like this.}
	
	\vfill
	
	{\scriptsize (take a look at the code to see how it works!)}

\end{frame}


\begin{frame}{The General Uncover Environment}
	If you want to step reveals of various elements in complex ways like this, use the \alert{uncover} environment along with the parameters we just covered:
	
	\vfill
	
	\begin{columns}[c]
	\column{0.5\textwidth}
	
		\uncover<4->{Tricksy text above appears last!}
	
		\bigskip
	
		\uncover<3->{This text is set to appear second, even though it's on the left.}
	
	\column{0.5\textwidth}
	
	\centering
	\begin{figure}
		\uncover<2->{
		\includegraphics[width=\columnwidth]{image6}
		\caption{Image appears first}
		}
	\end{figure}
	
	\end{columns}
	
	\vfill
	
	{\scriptsize (take a look at the code to see how it works!)}
	
\end{frame}


\begin{frame}{Overlays for Other Effects}

	We can \textcolor<2,5->{magenta}{change the colour} of text for one or more ``steps'' of a frame as we like
	
	\bigskip
	
	\uncover<3->{And similarly with other \textbf<4>{formatting commands.}}
	
	\bigskip
	
	\uncover<5->{In that case I \textrm<6->{nested} overlay parameters!} \uncover<7->{(and in this one!)}
	
	\vfill
	
	{\scriptsize (take a look at the code to see how it works!)}

\end{frame}


\subsection{Slide Transitions}

\begin{frame}{Animation in a PDF!}

\transglitter				%the transition command needs to be within the frame to which the effect is applied
						%list of transition effects is in section 14.3 of http://tug.ctan.org/macros/latex/contrib/beamer/doc/beameruserguide.pdf

	\begin{alertblock}{}
	\centering
		\textcolor{red}{Wasn't that \textbf{FANCY!!}}
	\end{alertblock}
	
	\bigskip
	
	Although you can do transition effects in Beamer, I highly recommend against it except under special circumstances. \pause
	
	\begin{itemize}
		\item Distracting for the audience, taking away from your content.
		\item Can look ``flashy'' and so unprofessional.
		\item Doesn't seem to work consistently in every pdf viewer (works in Acrobat Reader).
		\item Need to be in Full Screen mode.
	\end{itemize}

\end{frame}


\section{Talk Advice}


\begin{frame}{Timing is Everything}

	The best advice I can give you is\dots \pause
	
	\begin{alertblock}{}
		\centering{\textcolor{red}{Begin preparing early, and \textbf{PRACTICE!!}}\par
		}
	\end{alertblock} 
	 
	\bigskip \pause
	
	Although it's possible to use one file for both your paper and your talk, I didn't explain how to do it because I think it's a \alert{terrible} idea.
	
	\medskip
	
	Analogously, your paper \alert{isn't} a presentation---effective presentations need consideration independent of the origin paper.
	
	\bigskip \pause
	
	Consider the \textbf{amount of time} you have, distill the paper down into the key idea/contribution/observation to convey in that time. Then outline into sections and sub-sections, then fill-in the details.
	
\end{frame}


\begin{frame}

	Even for densely mathematical talks, think \textbf{conceptually} about what needs to be conveyed, and recognize that the audience might not all be specialists!
	
	\bigskip
	
	Be aware of how much information fits in a Frame, and how much detail the audience can absorb when they're also trying to build an overall conceptual picture of your work at the same time.
	
	\bigskip
	
	Use the question period to fill in details if asked! \pause
	
	\begin{block}{Last Tip!}
		Have some ``extra'' slides at the end of the presentation (after the Thanks slide) with the details that you can show if a question comes up. It'll make you look like an ultra-prepared superstar pro.
	\end{block}

\end{frame}


\begin{frame}							%I always like to thank my audience for attending---they took time from their busy lives to give me their
									%attention, consideration, and feedback. Be genuinely grateful for that!
	\centering
	{\Huge \bf Thanks!}
	
\end{frame}



\begin{frame}{Example of an Extra Details Slide!}		%there doesn't seem to be a way to hide these slides from the navigation ``bubbles'' along the top.
										%if anyone finds a way, I'd love to know about it!

	Since this is pure extra, I'd say feel free to make the text a little smaller as needed, and to cram stuff on here.
	
	\bigskip
	
	This is also a good place to include a \textbf{References} frame. References work just like they do in \LaTeX\ (using the \alert{cite} command and \alert{bibliography} environment on a frame) which is why I didn't cover them in this template or workshop. 

\end{frame}


\end{document}



%%%%%%%%%%%%%%%%%%%%%%%%%%%%%%%%%%%%%%%%%%%%%%%%%%%%%%%%%%%%%%%%%%
%%%%%%%%%%%%%%%%%%%%%%%%%%%%%%%%%%%%%%%%%%%%%%%%%%%%%%%%%%%%%%%%%%
%%
%%		If something isn't working out COMPILE AGAIN! (You have to compile THREE TIMES for TOC and Navigation to work!)
%%
%%		Beamer can get confused. Don't be afraid to Trash your AUX files and recompile fresh.
%%
%%		I've linked this several times already, but here's the full Beamer User Guide:
%%		http://tug.ctan.org/macros/latex/contrib/beamer/doc/beameruserguide.pdf
%%
%%		Hopefully this was helpful! As I said above, feel free to pull from this, distribute it, or modify it as you like.
%%		And GOOD LUCK with your own presentations!!
%%
%%%%%%%%%%%%%%%%%%%%%%%%%%%%%%%%%%%%%%%%%%%%%%%%%%%%%%%%%%%%%%%%%%
%%%%%%%%%%%%%%%%%%%%%%%%%%%%%%%%%%%%%%%%%%%%%%%%%%%%%%%%%%%%%%%%%%





